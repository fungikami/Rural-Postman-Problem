\documentclass[11pt]{article}

% -------------- PREÁMBULO ---------------

\usepackage[spanish]{babel} 
\usepackage[utf8]{inputenc} 
\usepackage{amsmath}
\usepackage{amssymb} 
\usepackage{graphicx}           % Incluir imágenes en LaTeX
\usepackage{color}              % Para colorear texto
\usepackage{subfigure}          % subfiguras
\usepackage{float}              % Podemos usar el especificador [H] en las figuras para que se queden donde queramos
\usepackage{capt-of}            % Permite usar etiquetas fuera de elementos flotantes (etiquetas de figuras)
\usepackage{sidecap}            % Para poner el texto de las imágenes al lado
\sidecaptionvpos{figure}{c}     % Para que el texto se alinee al centro vertical
\usepackage{caption}            % Para poder quitar numeracion de figuras
\usepackage{anysize}            % Para personalizar el ancho de  los márgenes
\marginsize{2cm}{2cm}{2cm}{2cm} % Izquierda, derecha, arriba, abajo
\usepackage{multicol}
\usepackage{multirow}
\setlength{\columnsep}{1cm}

% Para agregar encabezado y pie de página
\usepackage{fancyhdr} 
\usepackage{clrscode3e}

\usepackage{listings}
\pagestyle{fancy}
\fancyhf{}
\fancyhead[L]{\footnotesize USB}                  % Encabezado izquierda
\fancyhead[R]{\footnotesize Estructuras de datos en MIPS}  
\fancyfoot[R]{\footnotesize Informe}              % Pie derecha
\fancyfoot[C]{\thepage}                           % centro
\fancyfoot[L]{\footnotesize Ingeniería de la Computación}  %izquierda
\renewcommand{\footrulewidth}{0.4pt}

\newcommand{\coment}[1]{}
\definecolor{BurntOrange}{RGB}{247,148,42}

\begin{document}

% ----------------- PORTADA -----------------

\begin{center} 
   \newcommand{\HRule}{\rule{\linewidth}{0.5mm}}  

   \begin{minipage}{0.48\textwidth}
      \begin{center}
         \includegraphics[scale = 0.5]{logo.png}
      \end{center}
   \end{minipage}

   \vspace*{1.0cm}                       
   \textsc{\huge Universidad Simón Bolívar} \\ [1.5cm] 

   \begin{minipage}{0.9\textwidth} 
      \begin{center}                                                             
         \textsc{\LARGE Informe de Proyecto I }
      \end{center}
   \end{minipage} \\ [3cm]

   \vspace*{1cm}                                                                              
   \HRule \\ [0.4cm]                                                  
   {\huge \bfseries Estructuras de Datos en MIPS} \\ [0.4cm] 
   \HRule \\ [4cm]

   \begin{minipage}{\textwidth} 
      \begin{flushleft} \large    
         \textbf{\underline{Autor:}} \\ 
         Christopher Gómez (18-10892)\\
         Ka Fung (18-10492)\\
      \end{flushleft}
   \end{minipage}

   \begin{minipage}{\textwidth}    
      \vspace{-0.6cm}  
      \begin{flushright} \large    
         \textbf{\underline{Profesor:}} \\  
         Eduardo Blanco  
      \end{flushright}        
   \end{minipage} 

   \vspace*{1cm}
   \flushleft{\textbf{\Large Organización del Computador (CI3815)} }\\
   \vspace{2cm}  

   \begin{center} 
      {\large \today} 
   \end{center}     
\end{center}                                                      
                                                               
\newpage
                                                    
% -------------------------------------------

\section{Pseudocódigo}

Se presentan a continuación los pseudocódigos del programa que
se desea implementar. \\

Primeramente, se necesita extraer los datos necesarios de cada
archivo de entrada, para ello, se usan tablas de hash y listas.

\begin{codebox}
   \Procname{$\proc{Extraer-Datos}$}
   \li \id{archivoEst} = Leer archivo de estudiantes
   \li \id{tablaHashEst} = \kw{new} \proc{TablaHash}(101)

   \li \For linea in archivoEst:
   \li \Do
      \id{carnet} = Guardar carnet
      \li \id{nombre} = Guardar nombre
      \li \id{indice} = Guardar índice
      \li \id{creditosAprob} = Guardar número de créditos aprobados
      \li \id{est} = \kw{new} \proc{Estudiante}(\id{carnet}, \id{nombre}, \id{indice}, \id{creditosAprob})
      \li \id{tablaHashEst}.\proc{Insertar}(\id{carnet}, \id{est})
      \End

   \li
   \li \id{archivoMat} = Leer archivo de materias
   \li \id{tablaHashMat} = \kw{new} \proc{TablaHash}(101)
   \li \id{listaMat} = \kw{new} \proc{Lista}()
   \li \For linea in archivoMat:
   \li \Do
      \id{codigo} = Guardar codigo
      \li \id{nombre} = Guardar nombre
      \li \id{creditos} = Guardar creditos
      \li \id{numCupos} = Guardar número de cupos
      \li \id{minCreditos} = Guardar mínimo de créditos
      \li \id{mat} = \kw{new} \proc{Materia}(\id{codigo}, \id{nombre}, \id{creditos}, \id{numCupos}, \id{minCreditos})
      \li \id{tablaHashMat}.\proc{Insertar}(\id{carnet}, \id{est})
      \li \id{listaMat}.\proc{Insertar-Ordenado}(\id{codigo}, \id{f})
      \End

   \li
   \li \id{listaSol} = \kw{new} \proc{Lista}()
   \li \id{archivoSol} = Leer archivo de solicitudes
   \li \For linea in archivoSol:
   \li \Do
      \id{carnet} = Guardar carnet del estudiante
      \li \id{est} = \id{tablaHashEst}.\proc{Obtener-Valor}(\id{carnet})
      \li \id{codigo} = Guardar codigo de la materia
      \li \id{mat} = \id{tablaHashMat}.\proc{Obtener-Valor}(\id{codigo})
      \li \id{sol} = \kw{new} \proc{Solicitud}(\id{est}, \id{mat}, `S')
      \li \id{listaSol}.\proc{Insertar}(\id{sol}) \label{li:Extraer-Datos-final}
      \End
   \End
   \end{codebox}

   Al terminar este pseudocódigo, se debe tener una lista de solicitudes,
   una lista de códigos de materias en orden lexicográfico, una tabla de
   estudiantes, y una tabla de materias, la idea ahora es procesar la lista
   de solicitudes para que cada materia tenga una lista de estudiantes inscritos.

   \begin{codebox}
      \setlinenumberplus{li:Extraer-Datos-final}{1}
      \Procname{$\proc{Procesar-Solicitudes}$}
      \li \For sol in listaSol:
      \Do
         \li \id{est} = \attrib{sol}{estudiante}
         \li \id{mat} = \attrib{sol}{materia}
         \li \id{mat}.\proc{Agregar-Estudiante}(\id{est}) \label{li:Procesar-Solicitudes-final}
         \End
      \End
   \end{codebox}

   Ahora, cada materia contiene una lista con los estudiantes inscritos. Se
   asume que la estructura se encarga de mantener actualizado el número de
   cupos y de agregar en orden a los estudiantes en su lista de estudiantes.
   Así, para finalizar esta primera etapa solamente resta escribir en el
   archivo de salida cada materia con sus estudiantes inscritos; nótese que para
   estas inclusiones no se toma en cuenta el número de créditos aprobados del estudiante
   ni el mínimo de créditos requeridos por la materia, ya que la modalidad
   indica que en la primera etapa se aceptar \textbf{todas} las solicitudes.

   \begin{codebox}
      \setlinenumberplus{li:Procesar-Solicitudes-final}{1}
      \Procname{$\proc{Generar-Archivo-Tentativo}$}
      \li \id{archivoTen} = Abrir archivo tentativo a escribir
      \li \For mat in listaMat:
      \Do
         \li \id{archivoTen}.\proc{Escribir}(`$<$\attrib{mat}{codigo}$>$ ')
         \li \id{archivoTen}.\proc{Escribir}(```$<$\attrib{mat}{nombre}$>$'' ')
         \li \id{archivoTen}.\proc{Escribir}(`$<$\attrib{mat}{numCupos}$>\backslash$n')
         
         \li \For est in \attrib{mat}{estudiantes}:
         \Do
            \li \id{archivoTen}.\proc{Escribir}(`   $<$\attrib{est}{carnet}$>$ ')
            \li \id{archivoTen}.\proc{Escribir}(`$<$\attrib{est}{nombre}$>\backslash$n') \label{li:Generar-Archivo-Tentativo-final}
            \End
         \End
         \End
      \End
   \end{codebox}

   Luego, se procesan las solicitudes de corrección. Al terminar el siguiente pseudocódigo, 
   se debe tener una lista de solicitudes de corrección. Para procesar dicha lista, se comienza
   aceptando todas las eliminaciones y paralelamente creando una lista de solicitudes de inclusión
   que estará ordenada de acuerdo a la prioridad dada en la modalidad M1.5: se le da más prioridad a los 
   estudiantes con menor número de créditos aprobados. \\

   Una vez aceptadas todas las eliminaciones, se procede a liberar cupos de las materias (líneas
   64-70),
   para ello, primero se eliminan de cada materia a los estudiantes que no cuenten con los
   requisitos necesarios para cursarlas (número mínimo de créditos aprobados, en este caso),
   y luego se elimina de las materias que exceden el número de cupos a los estudiantes con 
   menor prioridad. Por último, se aceptan
   inclusiones siguiendo la lista de prioridad siempre y cuando no se exceda el número de cupos.

   \begin{codebox}
      \setlinenumberplus{li:Generar-Archivo-Tentativo-final}{1}
      \Procname{$\proc{Extraer-Datos-Correccion}$}
      \li \id{listaSolCor} = \kw{new} \proc{Lista}()
      \li \id{archivoCor} = Leer archivo de solicitudes de corrección
      \li \For linea in archivoCor:
      \li \Do
         \id{carnet} = Guardar carnet del estudiante
         \li \id{est} = \id{tablaHashEst}.\proc{Obtener-Valor}(\id{carnet})
         \li \id{codigo} = Guardar codigo de la materia
         \li \id{mat} = \id{tablaHashMat}.\proc{Obtener-Valor}(\id{codigo})
         \li \id{op} = Guardar operación de la solicitud
         \li \id{sol} = \kw{new} \proc{Solicitud}(\id{est}, \id{mat}, \id{op})
         \li \id{listaCor}.\proc{Insertar}(\id{sol}) \label{li:Extraer-Datos-Correccion-final}
         \End
      \End
   \end{codebox}

   \begin{codebox}
      \setlinenumberplus{li:Extraer-Datos-Correccion-final}{1}
      \Procname{$\proc{Procesar-Solicitudes-Correccion}$}
      \li \id{listaPrioridad} = \kw{new} \proc{Lista}()
      \li \For sol in listaCor:
      \Do
         \li \id{est} = \attrib{sol}{estudiante}
         \li \id{mat} = \attrib{sol}{materia}

         \li \If \attrib{sol}{op} == `E':
         \Do 
            \li \id{mat}.\proc{Eliminar-Estudiante}(\id{est}) \label{li:Procesar-Solicitudes-Correccion-corte}
         \end{codebox}

         \begin{codebox}
         \setlinenumberplus{li:Procesar-Solicitudes-Correccion-corte}{1}
         \li\Do \Do \Else:
            \li \id{listaPrioridad}.\proc{Insertar-Ordenado}(\id{sol}, \id{g})
         \li
         \End
         \End
      
      \li \For \id{mat} in listaMat:
      \Do
         \li \For \id{est} in \attrib{mat}{estudiantes}:
         \Do
            \li \If \attrib{est}{creditosAprob} $<$ \attrib{mat}{minCreditos}:
            \Do \li \id{mat}.\proc{Eliminar-Estudiante}(\id{est})
         \End
            \End
      \End

      \li \For \id{mat} in listaMat:
         \li \Do \While \attrib{mat}{numCupos} $<$ 0:
            \li \Do // Eliminar al estudiante con más créditos aprobados. 
         \End
         \li
      \End
      \li \For \id{sol} in \id{listaPrioridad}:
      \Do 
         \li \id{mat} = \attrib{sol}{materia}
         \li \id{est} = \attrib{sol}{estudiante}
         \li \If \attrib{mat}{cupos} $>$ 0:
         \Do
         \li \If \attrib{est}{creditosAprob} $\geq$ \attrib{mat}{minCreditos}:
         \Do 
         \li \id{mat}.\proc{Agregar-Estudiante}(\id{est})
         \label{li:Procesar-Solicitudes-Correccion-final}
         \End
      \End
   \end{codebox}

   Finalmente, se escribe en el archivo de salida cada materia con sus 
   estudiantes inscritos y eliminados. Si el estudiante fue inscrito en
   el proceso de inscripción, no se denota la operación. En cambio, si 
   estaba inscrito y fue eliminado, se denota la operación de eliminación,
   o si se inscribió en corrección, se denota la operación de inscripción.
   No estarán en en archivo definitivo los estudiantes cuya inclusión se
   haya negado en la corrección, bien sea por falta de créditos o porque
   no habían cupos disponibles.

   \begin{codebox}
      \setlinenumberplus{li:Procesar-Solicitudes-Correccion-final}{1}
      \Procname{$\proc{Generar-Archivo-Definitivo}$}
      \li \id{archivoDef} = Abrir archivo definitivo a escribir
      \li \For mat in listaMat:
      \Do
         \li \id{archivoDef}.\proc{Escribir}(`$<$\attrib{mat}{codigo}$>$ ')
         \li \id{archivoDef}.\proc{Escribir}(```$<$\attrib{mat}{nombre}$>$'' ')
         \li \id{archivoDef}.\proc{Escribir}(`$<$\attrib{mat}{numCupos}$>\backslash$n')
         
         \li \For est in \attrib{mat}{estudiantes}:
         \Do
            \li \id{archivoTen}.\proc{Escribir}(`$<$\attrib{est}{carnet}$>$ ')
            \li \id{archivoTen}.\proc{Escribir}(`$<$\attrib{est}{nombre}$>$ ') 
            \li \id{archivoTen}.\proc{Escribir}(`$<$\attrib{est}{op}$>\backslash$n') \label{li:Generar-Archivo-Definitivo-final}
            \End
         \End
         \End
      \End
   \end{codebox}

   \section{Estructuras utilizadas}

   En la seccion anterior se menciona el uso de distintas estructuras de datos
   utilizadas en el diseño del programa. En esta sección se enlistan cada una
   de ellas, junto con sus atributos y operaciones. En algunos casos, se implementaron
   más operaciones de las aquí mencionadas, que se dejaron en el código fuente.

   \begin{itemize}
      \item \proc{Par}:
      \begin{itemize}
         \item Atributos:

         \begin{itemize}
            \item Primer elemento.
            \item Segundo elemento.
         \end{itemize}
      \end{itemize}
      
      \pagebreak
      \begin{itemize}
         \item Operaciones:

         \begin{itemize}
            \item \proc{Crear}(\id{primero}, \id{segundo})
         \end{itemize}
      \end{itemize}

      \item \proc{Lista}:

      \begin{itemize}
         \item Atributos:

         \begin{itemize}
            \item Cabeza.
            \item Tamaño.
         \end{itemize}
      \end{itemize}

      \begin{itemize}
         \item Operaciones:

         \begin{itemize}
            \item \proc{Crear}()
            \item \proc{Insertar}(\id{elemento})
            \item \proc{Insertar-Ordenado}(\id{elemento}, \id{f})
         \end{itemize}
      \end{itemize}

      \item \proc{TablaHash}:
      
      \begin{itemize}
         \item Atributos:

         \begin{itemize}
            \item Tamaño.
            \item Tabla.
         \end{itemize}
      \end{itemize}

      \begin{itemize}
         \item Operaciones:
         
         \begin{itemize}
            \item \proc{Crear}(\id{tam})
            \item \proc{Insertar}(\id{clave}, \id{valor})
            \item \proc{Obtener-Valor}(\id{clave})
         \end{itemize}
      \end{itemize}

      \item \proc{Estudiante}:
      \begin{itemize}
         \item Atributos:

         \begin{itemize}
            \item Carnet.
            \item Nombre.
            \item Índice.
            \item Créditos aprobados.
         \end{itemize}
      \end{itemize}

      \begin{itemize}
         \item Operaciones:
         
         \begin{itemize}
            \item \proc{Crear}(\id{carnet}, \id{nombre}, \id{indice}, \id{creditosAprob})
         \end{itemize}
      \end{itemize}

      \item \proc{Materia}:
      \begin{itemize}
         \item Atributos:

         \begin{itemize}
            \item Código.
            \item Nombre.
            \item Número de créditos.
            \item Cupos.
            \item Número mínimo de créditos aprobados.
            \item Lista Estudiantes.
         \end{itemize}
      \end{itemize}
      
      \begin{itemize}
         \item Operaciones:
         
         \begin{itemize}
            \item \proc{Crear}(\id{codigo}, \id{nombre}, \id{creditos}, \id{numCupos}, \id{minCreditos})
            \item \proc{Aumentar-Cupo}()
            \item \proc{Disminuir-Cupo}()
            \item \proc{Agregar-Estudiante}(\id{Estudiante})
            \item \proc{Eliminar-Estudiante}(\id{Estudiante})
         \end{itemize}
      \end{itemize}


      \item \proc{Solicitud}:
      \begin{itemize}
         \item Atributos:

         \begin{itemize}
            \item Estudiante.
            \item Materia.
            \item Operación.
         \end{itemize}
      \end{itemize}

      \begin{itemize}
         \item Operaciones:
         
         \begin{itemize}
            \item \proc{Crear}(\id{est}, \id{mat}, \id{op})
         \end{itemize}
      \end{itemize}
      
   \end{itemize}

\section{Consideraciones}
Durante la implementación de las estructuras de datos, se decidió crear una tabla de hash para
almacenar las materias y los estudiantes, ya que se realizaba numerosas búsquedas durante el programa. 
Ejemplo de ello se presenta al procesar cada solicitud de inscripción en $\proc{Extraer-Datos}$
y $\proc{Extraer-Datos-Correccion}$, en la cual se tiene que buscar por cada solicitud, los datos
del estudiante y la lista de estudiantes de cada \proc{Materia} para insertar la inscripción o eliminación. 
Así, para evitar búsquedas lineales con estructuras como listas, con las tablas de hash implementadas se logra 
obtener la información de cada materia/estudiante en tiempo amortizado constante. En este sentido, la función 
de Hash escogida para \proc{TablaHash} fue basada en la función de Hash para Strings que utiliza el lenguaje 
de programación Java\footnote{https://cseweb.ucsd.edu/~kube/cls/100/Lectures/lec16/lec16-15.html}, la
cual es sencilla de calcular a la vez que da buenos resultados en la práctica. \\

Por otro lado, se implementaron distintas funciones de comparación (a las que refiere el pseudocódigo en
la sección 1 como $f$ y $g$ en las líneas 22 y 62). Para imprimir cada materia con sus
respectivos estudiantes inscritos y eliminados, se toma en cuenta el orden ascendente según el código
de la materia y el carnet del estudiante. Para ello, se realizan comparaciones por cada valor ASCII de 
un caracter del carnet al insertar el estudiante en la lista de estudiantes de la materia. 
En cambio, para tomar en cuenta la modalidad asignada en las solicitudes de corrección, se realiza una 
comparación entre el número de créditos aprobados de cada estudiante a inscribir. \\

Para la implementación y almacenamiento en MIPS de algunos atributos de las estructuras de datos
(como \proc{Estudiante} y \proc{Materia}), se realizaron funciones \proc{itoa} y \proc{atoi}, las cuales
convierten un entero a una cadena de caracteres y viceversa, lo cual facilita las comparaciones
numéricas cuando es necesario, y la impresión del archivo de texto al final de cada etapa. \\
\end{document}