\documentclass[11pt]{article}

% -------------- PREÁMBULO ---------------

\usepackage[spanish]{babel} 
\usepackage[utf8]{inputenc} 
\usepackage{amsmath}
\usepackage{amssymb} 
\usepackage{amsbsy} % librerias ams
\usepackage{graphicx}           % Incluir imágenes en LaTeX
\usepackage{color}              % Para colorear texto
\usepackage{subfigure}          % subfiguras
\usepackage{float}              % Podemos usar el especificador [H] en las figuras para que se queden donde queramos
\usepackage{capt-of}            % Permite usar etiquetas fuera de elementos flotantes (etiquetas de figuras)
\usepackage{sidecap}            % Para poner el texto de las imágenes al lado
\sidecaptionvpos{figure}{c}     % Para que el texto se alinee al centro vertical
\usepackage{caption}            % Para poder quitar numeracion de figuras
\usepackage{anysize}            % Para personalizar el ancho de  los márgenes
\marginsize{2cm}{2cm}{2cm}{2cm} % Izquierda, derecha, arriba, abajo
\usepackage{multicol}
\usepackage{multirow}
\setlength{\columnsep}{1cm}

% Para agregar encabezado y pie de página
\usepackage{fancyhdr} 
\usepackage{clrscode3e} % Para agregar pseudocódigo

\usepackage{listings}
\pagestyle{fancy}
\fancyhf{}
\fancyhead[L]{\footnotesize USB}                           % Encabezado izquierda
\fancyhead[R]{\footnotesize Algoritmo Heurístico RPP}      % Encabezado derecha
\fancyfoot[R]{\footnotesize Informe Proyecto II}                       % Pie derecha
\fancyfoot[C]{\thepage}                                    % Centro
\fancyfoot[L]{\footnotesize Ingeniería de la Computación}  % Izquierda
\renewcommand{\footrulewidth}{0.4pt}

\newcommand{\coment}[1]{}
\definecolor{BurntOrange}{RGB}{247,148,42}

\begin{document}

% ----------------- PORTADA -----------------

\begin{center} 
   \newcommand{\HRule}{\rule{\linewidth}{0.5mm}}  

   \begin{minipage}{0.48\textwidth}
      \begin{center}
         \includegraphics[scale = 0.8]{logotip.png}
      \end{center}
   \end{minipage}

   \vspace*{0.2cm}                       
   \textsc{\large Dpto. de Computación y Tecnología de la Información} \\ 
   \textsc{\large CI-2693 - Lab. de Algoritmos y Estructuras de Datos III} \\ [4cm] 

   \vspace*{1cm}                                                                              
   \HRule \\ [0.4cm]     
   \textsc{\Large Proyecto II } \\ [0.4cm]                                       
   {\Huge \bfseries Un algoritmo heurístico para resolver el Problema del Cartero Rural} \\ [0.4cm] 
   \HRule \\ [6cm]

   \begin{minipage}{\textwidth} 
      \begin{flushleft} \large    
         \textbf{\underline{Autor:}} \\ 
         Christopher Gómez (18-10892)\\
         Ka Fung (18-10492)\\
      \end{flushleft}
   \end{minipage}

   \begin{minipage}{\textwidth}    
      \vspace{-2cm}  
      \begin{flushright} \large    
         \textbf{\underline{Profesor:}} \\  
         Guillermo Palma  
      \end{flushright}        
   \end{minipage} \\ [2cm]

   \begin{center} 
      {\large \today} 
   \end{center}     
\end{center}                                                      
                                                               
\newpage
                                                    
% -------------------------------------------
\section{Introducción}

El siguiente estudio experimental consiste en implementar un algoritmo heurístico para
obtener soluciones aproximadas al Problema del Cartero Rural (RPP), modelado mediante el uso 
de grafos no dirigidos. Para ello, se empleará el algoritmo presentado por Pearn y Wu, el
cual es una modificación del algoritmo de Christofides et al. El RPP consiste en encontrar 
un ciclo de costo mínimo en un grafo no dirigido, tal que atraviese un conjunto de lados 
requeridos al menos una vez. \\

En el algoritmo de Pearn y Wu requiere de la implementación de un algoritmo de apareamiento 
perfecto de costo mínimo. Por ello, se implementará dos algoritmos heurísticos para obtener 
un apareamiento perfecto de un grafo completo (un algoritmo ávido y el Vertex-Scan presentado
por David Avis [1]), que proporcionan una solución aproximada y que sirven como una alternativa
mucho más eficiente y menos compleja que el algoritmo de Edmonds, el cual proporciona una 
solución exacta al problema. \\

Luego, el objetivo de este estudio consiste en comparar la eficacia y eficiencia de estos
dos algoritmos aplicados a solucionar 36 distintas del problema RPP. \\

\section{Diseño de la solución}

Para resolver el \emph{problema del cartero rural} se usó el algoritmo constructivo
proveído por el profesor del curso, basado en el presentado por Pearn y Wu en [2]. \\

La idea principal de dicho algoritmo consiste en modelar la instancia del problema
con un grafo no dirigido $G = (V, E)$ y un subconjunto $R \subseteq E$ que representa
las aristas por las que el cartero debe pasar en el ciclo buscado. Luego, construir a
partir del conjunto $R$ un grafo $G' = (V_R, R)$, al cual se le añaden lados y vértices en varias
ocasiones hasta obtener finalmente un grafo par y conexo del cual obtener un ciclo
euleriano que servirá como solución aproximada del problema. \\

Los procedimientos para añadir lados al grafo $G'$ en dos ocasiones
se pueden resumir y simplificar como los siguientes:

\begin{itemize}
   \item Se construye un grafo $G_t$ completo donde cada vértice
   es una componente conexa de $G'$ y el costo de cada lado corresponde
   al del camino de costo mínimo entre dichas componentes en $G$.
   
   \item Se obtiene el conjunto $E_{MST}$ de los lados del arbol mínimo
   cobertor de $G_t$.

   \item Se construye $E_t$ como el conjunto de lados de los
   caminos de costo mínimo asociados a cada lado de $E_{MST}$.

   \item Se agrega al conjunto de vértices de $G'$ los
   vértices que aparecen $E_t$, y luego los lados admitiendo
   duplicados.
\end{itemize}

En este punto se ha convertido $G'$ en un grafo conexo, pues
hemos añadido a cada componente conexa del grafo $G'$ inicial
los caminos más cortos que las unen, por lo que solo resta
hacer que $G'$ sea par para poder obtener de él un ciclo
euleriano:

\begin{itemize}
   \item Se obtiene el conjunto $V_0$ de los vértices de 
   grado impar de $G'$ y se construye de él un grafo completo
   $G_0$, donde el costo de cado lado corresponde al del camino
   de costo mínimo entre tales vértices en $G$.

   \item Se determina $M$ como el conjunto de lados que conforman
   un \textbf{apareamiento perfecto} de $G_0$
   
   \item Para cada lado de $M$, se obtiene el camino de costo
   mínimo asociado al mismo en $G$, y se añaden a $G'$ sus
   vértices y luego sus lados, admitiendo nuevamente los
   duplicados para estos últimos.
\end{itemize}

Se tiene que finalmente $G'$ es conexo y par, debido a que este
último procedimiento, al tratarse de los caminos obtenidos de
un apareamiento perfecto, añadirá un lado a los vértices inicial y final,
inicialmente de grado impar, y dos lados a todos los demás vértices
del camino, manteniéndolos pares. \\

Ya teniendo un grafo $G'$ conexo y par, se puede obtener de él
un ciclo euleriano que será la solución aproximada al problema 
modelado por $G$. \\

Cabe destacar que el algoritmo propuesto busca obtener un
\textbf{apareamiento perfecto de costo mínimo} para hacer el
grafo $G'$ par, sin embargo, para simplificar la implementación
se usaron algoritmos que obtienen un apareamiento perfecto 
aproximado al de costo mínimo. Además, es importante mencionar
que si luego de añadir los primeros lados a $G'$ el grafo resultante
es par, se obtiene directamente el ciclo euleriano, y análogamente
si al comienzo del algoritmo $G'$ era conexo y par.

\section{Detalles de la implementacion}

La implementación de todos los algoritmos se realizó en el
lenguaje de programación \texttt{Kotlin}, mediante el uso
y la modificación de la librería \texttt{grafoLib}, construida
a lo largo del curso. \\

Uno de los primeros aspectos a tomar en cuenta consistió en
modificar las implementaciones de la clase \texttt{GrafoNoDirigido}
para permitir tener lados duplicados, pues es un requerimiento
del algoritmo en dos ocasiones. \\

Luego, dado que los grafos de la librería \texttt{grafoLib}
tienen un número fijo $n$ de vértices en el intervalo $[0 .. n)$,
surgió el problema de cómo crear el grafo inicial $G'$, teniendo
que sus vértices consistirán en los vértices que aparecen en $R$,
que a pesar de ser un subconjunto de $V$, puede estar conformado,
por ejemplo, por los dos primeros y los dos últimos vértices de
$G$, generando un grafo de a lo sumo 4 vértices en el intervalo 
$[0..n)$, imposible en \texttt{grafoLib} si $n > 4$. \\

Para resolver este problema, en la implementación se construye
una función biyectiva $f: V \rightarrow V_R $ que es usada para
modelar $G'$ mediante un grafo isomorfo que permita relacionar
sus vértices con los vértices correspondientes de $G$
cuando sea necesario, como también relacionar $G$ con el grafo
isomorfo de $G'$ por medio de $f^{-1}$. Este detalle no afecta
gravemente la eficiencia de la implementación debido a que el 
modelado de $f$ y $f^{-1}$ se basa en arreglos dinámicos y 
diccionarios como tablas de hash, por lo que cada mapeo que
se hace con estas funciones ocurre en un tiempo amortizado
constante, \\

Análogamente, se presentó el mismo problema más adelante
con el grafo $G_0$ y el conjunto de vértices de grado impar,
solucionado de forma similar modelando $G_0$ mediante un grafo
isomorfo inducido por la función $h: V_R \rightarrow V_0 $ que
permite relacionar sus vértices con los vértices del isomorfismo
de $G'$. En el código fuente solo se construye $ h^{-1} $ por ser
la única necesaria. \\

En seguida, como el algoritmo requiere de la búsqueda de 
caminos de costo mínimo (al hallar caminos entre pares de componentes
conexas y de pares de vértices de $V_0$), se requirió modificar
el algoritmo de Dijkstra, originalmente pensado para dígrafos,
para hallar caminos de costo mínimo en grafos no dirigidos. 
La modificación, realizada en la clase \texttt{DijsktraGrafoNoDirigido},
consistió en dar ``orientación'' a las relajaciones de las aristas,
a pesar de que se tiene que $(u, v) = (v, u)$, con el fin de obtener
resultados correctos, además de modificaciones mínimas de la
misma índole al momento de hacer \emph{backtracking} para hallar
caminos de costo mínimo. \\

De igual manera, se implementó una modificación del algoritmo para
obtener ciclos eulerianos en la clase \texttt{CicloEulerianoGrafoNoDirigido},
con la misma idea de dar orientación a las aristas y que estas estén
orientadas de forma correcta al obtener el ciclo, lo cual agrega al
algoritmo un trabajo de tiempo lineal con respecto a los lados del
ciclo, sin afectar la complejidad temporal del algoritmo original. \\

De esta forma, para obtener el 
Algoritmo de Dijkstra para obtener los caminos de costo mínimo en grafos no dirigidos. \\

Algoritmo de Prim para obtener el árbol mínimo cobertor.\\

Algoritmo para obtener el ciclo euleriano de grafos no dirigidos.\\

Algoritmo de apareamiento perfecto ávido.\\

Algoritmo de apareamiento perfecto Vertex-Scan.\\

Programa cliente para la solución del problema RPP.\\


\section{Detalles de la plataforma}
Los siguientes son los detalles de la máquina y el entorno donde se ejecutaron las implementaciones:

\begin{itemize}
   \item \textbf{Sistema operativo}: GNU/Linux (Debian 11 Bullseye 64-bit).
   \item \textbf{Procesador}: Intel(R) Core(TM) i3-2120 CPU @3.30GHz.
   \item \textbf{Memoria RAM}: 4,00 GB (3,88 GB usables).
   \item \textbf{Compilador}: kotlinc-jvm 1.6.10 (JRE 11.0.12+7-post-Debian-2).
   \item \textbf{Entorno de ejecución}:OpenJDK Runtime Environment (build 11.0.12+7-post-Debian-2).
   % \item \textbf{Flags (máquina virtual)}: -Xmx2g.
\end{itemize}

\section{Resultados experimentales}

Para realizar una comparación entre la solución obtenida y la solución óptima de una instancias
del problema de RPP, se calculó el porcentaje de desviación con la fórmula:

\begin{equation}
    \%desv = \cfrac{\text{valor obtenido} - \text{valor óptimo}}{\text{valor óptimo}} *100
\end{equation}

\begin{table}[htbp]
   \begin{center}
   \begin{tabular}{|r|r|r|r|r|}
   \hline
   \multicolumn{1}{|c|}{Nombre de} & \multicolumn{1}{c|}{Valor} & \multicolumn{ 2}{c|}{Desviación (\%)} & \multicolumn{1}{c|}{Promedio } \\ \cline{3-4}
   \multicolumn{1}{|c|}{la instancia} & \multicolumn{1}{c|}{óptimo} & \multicolumn{1}{c|}{Alg. Ávido} & \multicolumn{1}{c|}{Vertex-Scan} & \multicolumn{1}{c|}{Resultado V-S} \\ \hline
   \texttt{UR132} & \textbf{23913} & 16,271 & 20,655 & 28852 \\
   \texttt{UR135} & \textbf{33088} & 14,335 & 16,814 & 38651 \\
   \texttt{UR137} & \textbf{42797} & 10,347 & 13,594 & 48615 \\
   \texttt{UR142} & \textbf{25548} & 15,935 & 19,035 & 30411 \\
   \texttt{UR145} & \textbf{39008} & 7,342 & 12,603 & 43924 \\
   \texttt{UR147} & \textbf{55959} & 6,346 & 7,604 & 60214 \\
   \texttt{UR152} & \textbf{28975} & 11,365 & 17,336 & 33998 \\
   \texttt{UR155} & \textbf{49156} & 4,669 & 7,078 & 52635 \\
   \texttt{UR157} & \textbf{70231} & 3,91 & 4,595 & 73458 \\
   \texttt{UR162} & \textbf{32341} & 10,998 & 12,462 & 36371 \\
   \texttt{UR165} & \textbf{58800} & 4,978 & 6,2 & 62446 \\
   \texttt{UR167} & \textbf{82481} & 2,917 & 3,572 & 85427 \\
   \texttt{UR532} & \textbf{17277} & 16,438 & 22,261 & 21123 \\
   \texttt{UR535} & \textbf{23635} & 16,306 & 16,839 & 27615 \\
   \texttt{UR537} & \textbf{30098} & 9,253 & 12,324 & 33807 \\
   \texttt{UR542} & \textbf{17830} & 14,173 & 20,269 & 21444 \\
   \texttt{UR545} & \textbf{29648} & 6,365 & 8,835 & 32267 \\
   \texttt{UR547} & \textbf{38692} & 4,784 & 7,877 & 41740 \\
   \texttt{UR552} & \textbf{20097} & 12,579 & 16,875 & 23488 \\
   \texttt{UR555} & \textbf{34488} & 4,126 & 8,069 & 37271 \\
   \texttt{UR557} & \textbf{48307} & 4,651 & 3,97 & 50225 \\
   \texttt{UR562} & \textbf{24556} & 7,188 & 10,541 & 27144 \\
   \texttt{UR565} & \textbf{42828} & 4,056 & 5,17 & 45042 \\
   \texttt{UR567} & \textbf{58971} & 4,087 & 4,015 & 61339 \\
   \texttt{UR732} & \textbf{21114} & 18,751 & 21,313 & 25614 \\
   \texttt{UR735} & \textbf{28663} & 10,623 & 15,26 & 33037 \\
   \texttt{UR737} & \textbf{36588} & 6,308 & 11,45 & 40777 \\
   \texttt{UR742} & \textbf{22557} & 13,078 & 17,916 & 26598 \\
   \texttt{UR745} & \textbf{32493} & 10,19 & 10,96 & 36054 \\
   \texttt{UR747} & \textbf{47764} & 5,481 & 6,385 & 50814 \\
   \texttt{UR752} & \textbf{25131} & 12,614 & 15,796 & 29101 \\
   \texttt{UR755} & \textbf{41774} & 4,517 & 8,357 & 45265 \\
   \texttt{UR757} & \textbf{58416} & 4,002 & 5,876 & 61849 \\
   \texttt{UR762} & \textbf{27880} & 8,691 & 13,155 & 31548 \\
   \texttt{UR765} & \textbf{50492} & 4,565 & 5,696 & 53368 \\
   \texttt{UR767} & \textbf{72950} & 3,413 & 3,911 & 75803 \\ \hline
   \end{tabular}
   \end{center}
   \caption{Desviación de los resultados obtenidos para cada algoritmo}
   \label{tablaDesv}
\end{table}   

\begin{table}[htbp]
   \begin{center}
   \begin{tabular}{|r|r|r|}
   \hline
   \multicolumn{1}{|c|}{Nombre de} & \multicolumn{ 2}{c|}{Promedio del tiempo (seg.)} \\ \cline{2-3}
   \multicolumn{1}{|c|}{la instancia} & \multicolumn{1}{c|}{Alg. Ávido} & \multicolumn{1}{c|}{Vertex-Scan} \\ \hline
   \texttt{UR132} & 0,673 & 0,637 \\
   \texttt{UR135} & 1,096 & 0,965 \\
   \texttt{UR137} & 1,217 & 1,05 \\
   \texttt{UR142} & 1,025 & 0,81 \\
   \texttt{UR145} & 1,164 & 1,058 \\
   \texttt{UR147} & 1,373 & 1,192 \\
   \texttt{UR152} & 1,169 & 0,929 \\
   \texttt{UR155} & 1,312 & 1,223 \\
   \texttt{UR157} & 1,352 & 1,169 \\
   \texttt{UR162} & 1,224 & 1,313 \\
   \texttt{UR165} & 1,375 & 1,215 \\
   \texttt{UR167} & 1,416 & 1,148 \\
   \texttt{UR532} & 0,319 & 0,296 \\
   \texttt{UR535} & 0,45 & 0,416 \\
   \texttt{UR537} & 0,47 & 0,467 \\
   \texttt{UR542} & 0,334 & 0,327 \\
   \texttt{UR545} & 0,529 & 0,457 \\
   \texttt{UR547} & 0,449 & 0,443 \\
   \texttt{UR552} & 0,388 & 0,391 \\
   \texttt{UR555} & 0,505 & 0,464 \\
   \texttt{UR557} & 0,516 & 0,445 \\
   \texttt{UR562} & 0,487 & 0,431 \\
   \texttt{UR565} & 0,516 & 0,501 \\
   \texttt{UR567} & 0,693 & 0,511 \\
   \texttt{UR732} & 0,492 & 0,479 \\
   \texttt{UR735} & 0,737 & 0,666 \\
   \texttt{UR737} & 0,822 & 0,746 \\
   \texttt{UR742} & 0,596 & 0,525 \\
   \texttt{UR745} & 0,802 & 0,688 \\
   \texttt{UR747} & 0,807 & 0,75 \\
   \texttt{UR752} & 0,647 & 0,693 \\
   \texttt{UR755} & 1,047 & 0,815 \\
   \texttt{UR757} & 0,909 & 0,866 \\
   \texttt{UR762} & 0,873 & 0,732 \\
   \texttt{UR765} & 1,001 & 0,968 \\
   \texttt{UR767} & 0,907 & 0,858 \\ \hline
   \multicolumn{1}{|c|}{\textbf{Total}} & 29,692 & 26,644 \\ \hline
   \multicolumn{1}{|c|}{\textbf{Promedio}} & 0,825 & 0,74 \\ \hline
   \end{tabular}
   \end{center}
   \caption{Tiempo de ejecución de la implementación para cada instancia}
   \label{tablaTiempos}
\end{table}
   

\section{Análisis de los resultados}


\section{Conclusiones}

\begin{itemize}
   \item 
   \item 
\end{itemize}

\section{Referencias (en caso de tenerlas)}

ñema. \\

\end{document}