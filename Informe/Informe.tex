\documentclass[11pt]{article}

% -------------- PREÁMBULO ---------------

\usepackage[spanish]{babel} 
\usepackage[utf8]{inputenc} 
\usepackage{amsmath}
\usepackage{amssymb} 
\usepackage{graphicx}           % Incluir imágenes en LaTeX
\usepackage{color}              % Para colorear texto
\usepackage{subfigure}          % subfiguras
\usepackage{float}              % Podemos usar el especificador [H] en las figuras para que se queden donde queramos
\usepackage{capt-of}            % Permite usar etiquetas fuera de elementos flotantes (etiquetas de figuras)
\usepackage{sidecap}            % Para poner el texto de las imágenes al lado
\sidecaptionvpos{figure}{c}     % Para que el texto se alinee al centro vertical
\usepackage{caption}            % Para poder quitar numeracion de figuras
\usepackage{anysize}            % Para personalizar el ancho de  los márgenes
\marginsize{2cm}{2cm}{2cm}{2cm} % Izquierda, derecha, arriba, abajo
\usepackage{multicol}
\usepackage{multirow}
\setlength{\columnsep}{1cm}

% Para agregar encabezado y pie de página
\usepackage{fancyhdr} 
\usepackage{clrscode3e} % Para agregar pseudocódigo

\usepackage{listings}
\pagestyle{fancy}
\fancyhf{}
\fancyhead[L]{\footnotesize USB}                           % Encabezado izquierda
\fancyhead[R]{\footnotesize Algoritmo Heurístico RPP}      % Encabezado derecha
\fancyfoot[R]{\footnotesize Informe}                       % Pie derecha
\fancyfoot[C]{\thepage}                                    % Centro
\fancyfoot[L]{\footnotesize Ingeniería de la Computación}  % Izquierda
\renewcommand{\footrulewidth}{0.4pt}

\newcommand{\coment}[1]{}
\definecolor{BurntOrange}{RGB}{247,148,42}

\begin{document}

% ----------------- PORTADA -----------------

\begin{center} 
   \newcommand{\HRule}{\rule{\linewidth}{0.5mm}}  

   \begin{minipage}{0.48\textwidth}
      \begin{center}
         \includegraphics[scale = 0.5]{logo.png}
      \end{center}
   \end{minipage}

   \vspace*{1.0cm}                       
   \textsc{\huge Universidad Simón Bolívar} \\ [1.5cm] 

   \begin{minipage}{0.9\textwidth} 
      \begin{center}                                                             
         \textsc{\LARGE Informe de Proyecto II }
      \end{center}
   \end{minipage} \\ [3cm]

   \vspace*{1cm}                                                                              
   \HRule \\ [0.4cm]                                                  
   {\huge \bfseries Un algoritmo heurístico para resolver el Problema del Cartero Rural} \\ [0.4cm] 
   \HRule \\ [4cm]

   \begin{minipage}{\textwidth} 
      \begin{flushleft} \large    
         \textbf{\underline{Autor:}} \\ 
         Christopher Gómez (18-10892)\\
         Ka Fung (18-10492)\\
      \end{flushleft}
   \end{minipage}

   \begin{minipage}{\textwidth}    
      \vspace{-0.6cm}  
      \begin{flushright} \large    
         \textbf{\underline{Profesor:}} \\  
         Guillermo Palma  
      \end{flushright}        
   \end{minipage} 

   \vspace*{1cm}
   \flushleft{\textbf{\Large Algoritmos y Estructuras III (CI2693)} }\\
   \vspace{2cm}  

   \begin{center} 
      {\large \today} 
   \end{center}     
\end{center}                                                      
                                                               
\newpage
                                                    
% -------------------------------------------
\section{Introducción}

El siguiente estudio experimental consiste en ejecutar un algoritmo heurístico para
obtener soluciones aproximadas al Problema del Cartero Rural, modelado mediante el uso de
grafos no dirigidos. Para ello, se implentará el algoritmo presentado por Pearn y Wu, el
cual es una modificación del algoritmo de Christofides et al. \\

Se compararán dos algoritmos para la obtención de apareamientos perfectos. \\

Los siguientes son los detalles de la máquina y el entorno donde se ejecutaron los algoritmos:

\begin{itemize}
   \item \textbf{Sistema operativo}: GNU/Linux (Linux Mint 20 Ulyana).
   \item \textbf{Procesador}: Intel(R) Core(TM) i7-8750H.
   \item \textbf{Memoria RAM}: 8GB.
   \item \textbf{Compilador}: Kotlin version 1.5.31 (JRE 11.0.9.1+1-Ubuntu-0ubuntu1.20.04).
   \item \textbf{Entorno de ejecución}: OpenJDK Runtime Environment (11.0.9.1).
   % \item \textbf{Flags (máquina virtual)}: -Xmx2g.
\end{itemize}

\section{Diseño de la solución}

hola. \\


\section{Detalles de la implementacion}

chau. \\


\section{Datos de la plataforma}

hola. \\

\section{Resultados experimentales}

hola otra vez, aquí van dos tablas. \\

La primera tabla debe tener los valores obtenidos en la solución de las instancias de
RPP. En especifico la tabla debe mostrar:

1. El nombre de la instancia,

2. El valor óptimo de la instancia,

3. El porcentaje de desviación de la solución obtenida usando la heurística ávida del
Algoritmo 2.
4. El porcentaje de desviación de la solución obtenida usando la heurística Vertex-Scan
del Algoritmo 3

Para el caso de la heurística Vertex-Scan, para cada instancia debe presentar el promedio
de las soluciones de tres corridas.
La segunda tabla debe mostrar el tiempo promedio que tomó la ejecución de todas las
instancias, usando heurística ávida y la heurística Vertex-Scan.
\section{Análisis de los resultados}


\section{Conclusiones}

\begin{itemize}
   \item 
   \item 
\end{itemize}

\section{Referencias (en caso de tenerlas)}

ñema. \\

\end{document}