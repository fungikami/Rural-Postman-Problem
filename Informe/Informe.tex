\documentclass[11pt]{article}

% -------------- PREÁMBULO ---------------

\usepackage[spanish]{babel} 
\usepackage[utf8]{inputenc} 
\usepackage{amsmath}
\usepackage{amssymb} 
\usepackage{amsbsy} % librerias ams
\usepackage{graphicx}           % Incluir imágenes en LaTeX
\usepackage{color}              % Para colorear texto
\usepackage{subfigure}          % subfiguras
\usepackage{float}              % Podemos usar el especificador [H] en las figuras para que se queden donde queramos
\usepackage{capt-of}            % Permite usar etiquetas fuera de elementos flotantes (etiquetas de figuras)
\usepackage{sidecap}            % Para poner el texto de las imágenes al lado
\sidecaptionvpos{figure}{c}     % Para que el texto se alinee al centro vertical
\usepackage{caption}            % Para poder quitar numeracion de figuras
\usepackage{anysize}            % Para personalizar el ancho de  los márgenes
\marginsize{2cm}{2cm}{2cm}{2cm} % Izquierda, derecha, arriba, abajo
\usepackage{multicol}
\usepackage{multirow}
\setlength{\columnsep}{1cm}

% Para agregar encabezado y pie de página
\usepackage{fancyhdr} 
\usepackage{clrscode3e} % Para agregar pseudocódigo

\usepackage{listings}
\pagestyle{fancy}
\fancyhf{}
\fancyhead[L]{\footnotesize USB}                           % Encabezado izquierda
\fancyhead[R]{\footnotesize Algoritmo Heurístico RPP}      % Encabezado derecha
\fancyfoot[R]{\footnotesize Informe Proyecto II}                       % Pie derecha
\fancyfoot[C]{\thepage}                                    % Centro
\fancyfoot[L]{\footnotesize Ingeniería de la Computación}  % Izquierda
\renewcommand{\footrulewidth}{0.4pt}

\newcommand{\coment}[1]{}
\definecolor{BurntOrange}{RGB}{247,148,42}

\begin{document}

% ----------------- PORTADA -----------------

\begin{center} 
   \newcommand{\HRule}{\rule{\linewidth}{0.5mm}}  

   \begin{minipage}{0.48\textwidth}
      \begin{center}
         \includegraphics[scale = 0.8]{logotip.png}
      \end{center}
   \end{minipage}

   \vspace*{0.2cm}                       
   \textsc{\large Dpto. de Computación y Tecnología de la Información} \\ 
   \textsc{\large CI-2693 - Lab. de Algoritmos y Estructuras de Datos III} \\ [4cm] 

   \vspace*{1cm}                                                                              
   \HRule \\ [0.4cm]     
   \textsc{\Large Proyecto II } \\ [0.4cm]                                       
   {\Huge \bfseries Un algoritmo heurístico para resolver el Problema del Cartero Rural} \\ [0.4cm] 
   \HRule \\ [6cm]

   \begin{minipage}{\textwidth} 
      \begin{flushleft} \large    
         \textbf{\underline{Autor:}} \\ 
         Christopher Gómez (18-10892)\\
         Ka Fung (18-10492)\\
      \end{flushleft}
   \end{minipage}

   \begin{minipage}{\textwidth}    
      \vspace{-2cm}  
      \begin{flushright} \large    
         \textbf{\underline{Profesor:}} \\  
         Guillermo Palma  
      \end{flushright}        
   \end{minipage} \\ [2cm]

   \begin{center} 
      {\large \today} 
   \end{center}     
\end{center}                                                      
                                                               
\newpage
                                                    
% -------------------------------------------
\section{Introducción}

El siguiente estudio experimental consiste en implementar un algoritmo heurístico para
obtener soluciones aproximadas al Problema del Cartero Rural (RPP), modelado mediante el uso 
de grafos no dirigidos. Para ello, se empleará el algoritmo presentado por Pearn y Wu, el
cual es una modificación del algoritmo de Christofides et al. El RPP consiste en encontrar 
un ciclo de costo mínimo en un grafo no dirigido, tal que atraviese un conjunto de lados 
requeridos al menos una vez. \\

En el algoritmo de Pearn y Wu requiere de la implementación de un algoritmo de apareamiento 
perfecto de costo mínimo. Por ello, se implementará dos algoritmos heurísticos para obtener 
un apareamiento perfecto de un grafo completo (un algoritmo ávido y el Vertex-Scan presentado
por David Avis [1]), que proporcionan una solución aproximada y que sirven como una alternativa
mucho más eficiente y menos compleja que el algoritmo de Edmonds, el cual proporciona una 
solución exacta al problema. \\

Luego, el objetivo de este estudio consiste en comparar la eficacia y eficiencia de estos
dos algoritmos aplicados a solucionar 36 distintas del problema RPP. \\

\section{Diseño de la solución}

Construcción de grafo GR y G' (mapeo G -> GR) \\

Construcción del grafo G0 (mapeo GR -> G0) \\

Implementación del algoritmo de Dijkstra para grafos no dirigidos\\

Implementación del ciclo euleriano para grafos no dirigidos\\




\section{Detalles de la implementacion}

Algoritmo para obtener las componentes conexas en grafos no dirigidos. \\

Algoritmo de Dijkstra para obtener los caminos de costo mínimo en grafos no dirigidos. \\

Algoritmo de Prim para obtener el árbol mínimo cobertor.\\

Algoritmo para obtener el ciclo euleriano de grafos no dirigidos.\\

Algoritmo de apareamiento perfecto ávido.\\

Algoritmo de apareamiento perfecto Vertex-Scan.\\

Programa cliente para la solución del problema RPP.\\


\section{Detalles de la plataforma}
Los siguientes son los detalles de la máquina y el entorno donde se ejecutaron los algoritmos:

\begin{itemize}
   \item \textbf{Sistema operativo}: GNU/Linux (Linux Mint 20 Ulyana).
   \item \textbf{Procesador}: Intel(R) Core(TM) i7-8750H.
   \item \textbf{Memoria RAM}: 8GB.
   \item \textbf{Compilador}: Kotlin version 1.5.31 (JRE 11.0.9.1+1-Ubuntu-0ubuntu1.20.04).
   \item \textbf{Entorno de ejecución}: OpenJDK Runtime Environment (11.0.9.1).
   % \item \textbf{Flags (máquina virtual)}: -Xmx2g.
\end{itemize}

\section{Resultados experimentales}

Para realizar una comparación entre la solución obtenida y la solución óptima de una instancias
del problema de RPP, se calculó el porcentaje de desviación con la fórmula:

\begin{equation}
    \%desv = \cfrac{\text{valor obtenido} - \text{valor óptimo}}{\text{valor óptimo}} *100
\end{equation}


\begin{table}[htbp]
   \begin{center}
   \begin{tabular}{|r|r|r|r|}
   \hline
   \multicolumn{1}{|c|}{Nombre de} & \multicolumn{1}{c|}{Valor} & \multicolumn{ 2}{c|}{Desviación (\%) } \\ \cline{3-4}
   \multicolumn{1}{|c|}{la instancia} & \multicolumn{1}{c|}{óptimo} & \multicolumn{1}{c|}{Alg. Ávido} & \multicolumn{1}{c|}{Vertex-Scan} \\ \hline
   \texttt{UR132} & \textbf{23913} & 16,271 & 20,655 \\ 
   \texttt{UR135} & \textbf{33088} & 14,335 & 16,814 \\ 
   \texttt{UR137} & \textbf{42797} & 10,347 & 13,594 \\ 
   \texttt{UR142} & \textbf{25548} & 15,935 & 19,035 \\ 
   \texttt{UR145} & \textbf{39008} &  7,342 & 12,603 \\ 
   \texttt{UR147} & \textbf{55959} &  6,346 &  7,604 \\ 
   \texttt{UR152} & \textbf{28975} & 11,365 & 17,336 \\ 
   \texttt{UR155} & \textbf{49156} &  4,669 &  7,078 \\ 
   \texttt{UR157} & \textbf{70231} &   3,91 &  4,595 \\ 
   \texttt{UR162} & \textbf{32341} & 10,998 & 12,462 \\ 
   \texttt{UR165} & \textbf{58800} &  4,978 &    6,2 \\ 
   \texttt{UR167} & \textbf{82481} &  2,917 & 3,572 \\ 
   \texttt{UR532} & \textbf{17277} & 16,438 & 22,261 \\ 
   \texttt{UR535} & \textbf{23635} & 16,306 & 16,839 \\ 
   \texttt{UR537} & \textbf{30098} &  9,253 & 12,324 \\ 
   \texttt{UR542} & \textbf{17830} & 14,173 & 20,269 \\ 
   \texttt{UR545} & \textbf{29648} &  6,365 & 8,835 \\ 
   \texttt{UR547} & \textbf{38692} &  4,784 & 7,877 \\ 
   \texttt{UR552} & \textbf{20097} & 12,579 & 16,875 \\ 
   \texttt{UR555} & \textbf{34488} &  4,126 & 8,069 \\ 
   \texttt{UR557} & \textbf{48307} &  4,651 & 3,97 \\ 
   \texttt{UR562} & \textbf{24556} &  7,188 & 10,541 \\ 
   \texttt{UR565} & \textbf{42828} &  4,056 & 5,17 \\ 
   \texttt{UR567} & \textbf{58971} &  4,087 & 4,015 \\ 
   \texttt{UR732} & \textbf{21114} & 18,751 & 21,313 \\ 
   \texttt{UR735} & \textbf{28663} & 10,623 & 15,26 \\ 
   \texttt{UR737} & \textbf{36588} &  6,308 & 11,45 \\ 
   \texttt{UR742} & \textbf{22557} & 13,078 & 17,916 \\ 
   \texttt{UR745} & \textbf{32493} &  10,19 & 10,96 \\ 
   \texttt{UR747} & \textbf{47764} &  5,481 & 6,385 \\ 
   \texttt{UR752} & \textbf{25131} & 12,614 & 15,796 \\ 
   \texttt{UR755} & \textbf{41774} &  4,517 & 8,357 \\ 
   \texttt{UR757} & \textbf{58416} &  4,002 & 5,876 \\ 
   \texttt{UR762} & \textbf{27880} &  8,691 & 13,155 \\ 
   \texttt{UR765} & \textbf{50492} &  4,565 & 5,696 \\ 
   \texttt{UR767} & \textbf{72950} &  3,413 & 3,911 \\ \hline
   \end{tabular}
   \end{center}
   \caption{Desviación de los resultados obtenidos para cada algoritmo}
   \label{tablaDesv}
   \end{table}

\begin{table}[htbp]
   \begin{center}
   \begin{tabular}{|r|r|r|}
   \hline
   \multicolumn{1}{|c|}{Nombre de} & \multicolumn{ 2}{c|}{Promedio del tiempo (seg.)} \\ \cline{2-3}
   \multicolumn{1}{|c|}{la instancia} & \multicolumn{1}{c|}{Alg. Ávido} & \multicolumn{1}{c|}{Vertex-Scan} \\ \hline
   \texttt{UR132} & 0,673 & 0,637 \\
   \texttt{UR135} & 1,096 & 0,965 \\
   \texttt{UR137} & 1,217 & 1,05 \\
   \texttt{UR142} & 1,025 & 0,81 \\
   \texttt{UR145} & 1,164 & 1,058 \\
   \texttt{UR147} & 1,373 & 1,192 \\
   \texttt{UR152} & 1,169 & 0,929 \\
   \texttt{UR155} & 1,312 & 1,223 \\
   \texttt{UR157} & 1,352 & 1,169 \\
   \texttt{UR162} & 1,224 & 1,313 \\
   \texttt{UR165} & 1,375 & 1,215 \\
   \texttt{UR167} & 1,416 & 1,148 \\
   \texttt{UR532} & 0,319 & 0,296 \\
   \texttt{UR535} & 0,45 & 0,416 \\
   \texttt{UR537} & 0,47 & 0,467 \\
   \texttt{UR542} & 0,334 & 0,327 \\
   \texttt{UR545} & 0,529 & 0,457 \\
   \texttt{UR547} & 0,449 & 0,443 \\
   \texttt{UR552} & 0,388 & 0,391 \\
   \texttt{UR555} & 0,505 & 0,464 \\
   \texttt{UR557} & 0,516 & 0,445 \\
   \texttt{UR562} & 0,487 & 0,431 \\
   \texttt{UR565} & 0,516 & 0,501 \\
   \texttt{UR567} & 0,693 & 0,511 \\
   \texttt{UR732} & 0,492 & 0,479 \\
   \texttt{UR735} & 0,737 & 0,666 \\
   \texttt{UR737} & 0,822 & 0,746 \\
   \texttt{UR742} & 0,596 & 0,525 \\
   \texttt{UR745} & 0,802 & 0,688 \\
   \texttt{UR747} & 0,807 & 0,75 \\
   \texttt{UR752} & 0,647 & 0,693 \\
   \texttt{UR755} & 1,047 & 0,815 \\
   \texttt{UR757} & 0,909 & 0,866 \\
   \texttt{UR762} & 0,873 & 0,732 \\
   \texttt{UR765} & 1,001 & 0,968 \\
   \texttt{UR767} & 0,907 & 0,858 \\ \hline
   \multicolumn{1}{|c|}{\textbf{Total}} & 29,692 & 26,644 \\ \hline
   \multicolumn{1}{|c|}{\textbf{Promedio}} & 0,825 & 0,74 \\ \hline
   \end{tabular}
   \end{center}
   \caption{Tiempo de ejecución de la implementación para cada instancia}
   \label{tablaTiempos}
   \end{table}
   

\section{Análisis de los resultados}


\section{Conclusiones}

\begin{itemize}
   \item 
   \item 
\end{itemize}

\section{Referencias (en caso de tenerlas)}

ñema. \\

\end{document}